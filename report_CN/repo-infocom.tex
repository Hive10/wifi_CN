
%% bare_conf.tex
%% V1.3
%% 2007/01/11
%% by Michael Shell
%% See:
%% http://www.michaelshell.org/
%% for current contact information.
%%
%% This is a skeleton file demonstrating the use of IEEEtran.cls
%% (requires IEEEtran.cls version 1.7 or later) with an IEEE conference paper.
%%
%% Support sites:
%% http://www.michaelshell.org/tex/ieeetran/
%% http://www.ctan.org/tex-archive/macros/latex/contrib/IEEEtran/
%% and
%% http://www.ieee.org/

%%*************************************************************************
%% Legal Notice:
%% This code is offered as-is without any warranty either expressed or
%% implied; without even the implied warranty of MERCHANTABILITY or
%% FITNESS FOR A PARTICULAR PURPOSE!
%% User assumes all risk.
%% In no event shall IEEE or any contributor to this code be liable for
%% any damages or losses, including, but not limited to, incidental,
%% consequential, or any other damages, resulting from the use or misuse
%% of any information contained here.
%%
%% All comments are the opinions of their respective authors and are not
%% necessarily endorsed by the IEEE.
%%
%% This work is distributed under the LaTeX Project Public License (LPPL)
%% ( http://www.latex-project.org/ ) version 1.3, and may be freely used,
%% distributed and modified. A copy of the LPPL, version 1.3, is included
%% in the base LaTeX documentation of all distributions of LaTeX released
%% 2003/12/01 or later.
%% Retain all contribution notices and credits.
%% ** Modified files should be clearly indicated as such, including  **
%% ** renaming them and changing author support contact information. **
%%
%% File list of work: IEEEtran.cls, IEEEtran_HOWTO.pdf, bare_adv.tex,
%%                    bare_conf.tex, bare_jrnl.tex, bare_jrnl_compsoc.tex
%%*************************************************************************

% *** Authors should verify (and, if needed, correct) their LaTeX system  ***
% *** with the testflow diagnostic prior to trusting their LaTeX platform ***
% *** with production work. IEEE's font choices can trigger bugs that do  ***
% *** not appear when using other class files.                            ***
% The testflow support page is at:
% http://www.michaelshell.org/tex/testflow/



% Note that the a4paper option is mainly intended so that authors in
% countries using A4 can easily print to A4 and see how their papers will
% look in print - the typesetting of the document will not typically be
% affected with changes in paper size (but the bottom and side margins will).
% Use the testflow package mentioned above to verify correct handling of
% both paper sizes by the user's LaTeX system.
%
% Also note that the "draftcls" or "draftclsnofoot", not "draft", option
% should be used if it is desired that the figures are to be displayed in
% draft mode.
%
\documentclass[conference]{IEEEtran}

\usepackage{url, fancyvrb, framed, multirow, tabularx, graphicx, epstopdf, enumerate, array, cite, algorithmic, fixltx2e}

\usepackage[cmex10]{amsmath}
%\usepackage{breqn}

% correct bad hyphenation here
\hyphenation{op-tical net-works semi-conduc-tor}

\begin{document}
%
% paper title
% can use linebreaks \\ within to get better formatting as desired
% \title{A Real-time Gesture Recognition System Using Wireless Signals}
\title{Multiple Applications of Wireless Signals Using Channel State Information (CSI)}

% author names and affiliations
% use a multiple column layout for up to three different
% affiliations
%\author{\IEEEauthorblockN{Michael Shell}
%\IEEEauthorblockA{School of Electrical and\\Computer Engineering\\
%Georgia Institute of Technology\\
%Atlanta, Georgia 30332--0250\\
%Email: http://www.michaelshell.org/contact.html}
%\and
%\IEEEauthorblockN{Homer Simpson}
%\IEEEauthorblockA{Twentieth Century Fox\\
%Springfield, USA\\
%Email: homer@thesimpsons.com}
%\and
%\IEEEauthorblockN{James Kirk\\ and Montgomery Scott}
%\IEEEauthorblockA{Starfleet Academy\\
%San Francisco, California 96678-2391\\
%Telephone: (800) 555--1212\\
%Fax: (888) 555--1212}}

% conference papers do not typically use \thanks and this command
% is locked out in conference mode. If really needed, such as for
% the acknowledgment of grants, issue a \IEEEoverridecommandlockouts
% after \documentclass

% for over three affiliations, or if they all won't fit within the width
% of the page, use this alternative format:
%
%\author{\IEEEauthorblockN{Michael Shell\IEEEauthorrefmark{1},
%Homer Simpson\IEEEauthorrefmark{2},
%James Kirk\IEEEauthorrefmark{3},
%Montgomery Scott\IEEEauthorrefmark{3} and
%Eldon Tyrell\IEEEauthorrefmark{4}}
%\IEEEauthorblockA{\IEEEauthorrefmark{1}School of Electrical and Computer Engineering\\
%Georgia Institute of Technology,
%Atlanta, Georgia 30332--0250\\ Email: see http://www.michaelshell.org/contact.html}
%\IEEEauthorblockA{\IEEEauthorrefmark{2}Twentieth Century Fox, Springfield, USA\\
%Email: homer@thesimpsons.com}
%\IEEEauthorblockA{\IEEEauthorrefmark{3}Starfleet Academy, San Francisco, California 96678-2391\\
%Telephone: (800) 555--1212, Fax: (888) 555--1212}
%\IEEEauthorblockA{\IEEEauthorrefmark{4}Tyrell Inc., 123 Replicant Street, Los Angeles, California 90210--4321}}
\author{
\IEEEauthorblockN{Yuning Mao}
\IEEEauthorblockA{
Shanghai Jiao Tong University\\
Dongchuan Road 800, Shanghai, China\\
morningmoni@sjtu.edu.cn}
\and
\IEEEauthorblockN{Yuting Jia}
\IEEEauthorblockA{
Shanghai Jiao Tong University\\
Dongchuan Road 800, Shanghai, China\\
XXX@sjtu.edu.cn}
\and
\IEEEauthorblockN{Zhenfeng Shi}
\IEEEauthorblockA{
Shanghai Jiao Tong University\\
Dongchuan Road 800, Shanghai, China\\
XXX@sjtu.edu.cn}
}

% use for special paper notices
%\IEEEspecialpapernotice{(Invited Paper)}


% make the title area
\maketitle


\begin{abstract}
Wireless signals (e.g., WiFi) are almost everywhere nowadays. However, the research of wireless signals using channel state information (CS) has just started. In this project, we try several applications by leveraging the channel state information (CSI). We implement an end-to-end system which can be used to control electronic devices (e.g., laptop) by gesture recognition. We do experiments of indoor localization  using CSI. We also try using information of wireless signals to detect indoor human activities. Our system leverages the WiFi signals using off-the-shelf network interface card (Intel 5300) and achieves an average accuracy of XXX \% for a classification of 4? typical gestures.   

\end{abstract}
% IEEEtran.cls defaults to using nonbold math in the Abstract.
% This preserves the distinction between vectors and scalars. However,
% if the conference you are submitting to favors bold math in the abstract,
% then you can use LaTeX's standard command \boldmath at the very start
% of the abstract to achieve this. Many IEEE journals/conferences frown on
% math in the abstract anyway.

% no keywords

\begin{keywords}
Gesture Recognition, Indoor Localization, Wireless Signals, Signal Processing, Classification
\end{keywords}


% For peer review papers, you can put extra information on the cover
% page as needed:
% \ifCLASSOPTIONpeerreview
% \begin{center} \bfseries EDICS Category: 3-BBND \end{center}
% \fi
%
% For peerreview papers, this IEEEtran command inserts a page break and
% creates the second title. It will be ignored for other modes.
\IEEEpeerreviewmaketitle

\section{Introduction}
As the users shift from traditional PCs to mobile devices and their expectation for new methods of interaction increases, there are increasing demands for novel human-computer interfaces (HCI) through which the users can control various applications.
Among those recently proposed methods, gesture recognition has gained much popularity. There are now successful commercial devices such as XBOX Kinect and Leap Motion. 
However, the cost of these commercial devices is relatively high and it also takes efforts to install and set up them.
In addition, these devices only support in-sight gesture recognition since they are vision-based.
Researchers also developed ways to move the sensors onto the body so as to bolster out-of-sight scenarios, and yet wearing sensors itself is inconvenient and infeasible in many cases.

Nowadays, wireless signals are almost everywhere. They are used for communication, remote control and so on. However, the research of doing gesture recognition by leveraging wireless signals just started in 2013. It partially results from the fact that there are plenty of challenges when it comes to the use of wireless signals. 
Wireless signals are usually unstable and sensitive to environmental changes, there could be issues because of medium contention, multipath interference, noises caused by other objects and so on.
collecting wireless signals and processing them is hence challenging.
Recently proposed WiFi-based systems are based on analyzing the changes caused by human motion in the characteristics of the wireless signals, such as the received signal strength indicator (RSSI) or detailed channel state information (CSI). HOWEVER

In this paper, we propose an end-to-end gesture recognition system and implement a demo using off-the-shelf network interface card (Intel 5300) and \emph{CSI Tools} \cite{halperin2011tool}. 
Instead of collecting data first and analyzing them later on and then reporting on the average accuracy of classification, we aim to provide a real-time interactive system which can be used in real scenarios. 
The respond time of our system is as short as XXX. 
In addition, we use machine learning techniques to support classification of different gestures and the accuracy of the classification can hence increase as the users provide more samples.
We also support user-defined gestures. All the users need is to perform a gesture several times and then the system can store the pattern and recognize the gesture afterwards.

The rest of this paper is organized as follows.  The background is introduced in Section \ref{section-background}. The system design is illustrated in Section \ref{section-design}. Section \ref{section-implementation} and Section \ref{section-evaluation} demonstrates the example implementation and evaluation, respectively. The discussion of benefits and tradeoffs are presented in Section \ref{section-discussion}. Section \ref{section-conclusion} concludes the paper.

\section{Background} \label{section-background}

\subsection{Channel State Information (CSI)}
Modern WiFi devices that support IEEE 802.11n/ac standard typically consist of multiple transmit and multiple receive antennas and thus support MIMO. Each MIMO channel between each transmit-receive (TX-RX) antenna pair of a transmitter and receiver comprises of multiple subcarriers. These WiFi devices continuously monitor the state of the wireless channel to effectively perform transmit power allocations and rate adaptations for each individual MIMO stream such that the available capacity of the wireless channel is maximally utilized \cite{halperin2010802}. These devices quantify the state of the channel in terms of CSI values. The CSI values essentially characterize the Channel Frequency Response (CFR) for each subcarrier between each transmit-receive (TX-RX) antenna pair.
Let $M_T$ denote the number of transmit antennas, $M_R$ denote the number of receive antennas and $S_c$ denote the number of OFDM subcarriers. Let $X_i$ and $Y_i$ represent the $M_T$ dimensional transmitted signal vector and $M_R$ dimensional received signal vector, respectively, for subcarrier i and let Ni represent an $M_R$ dimensional noise vector. An $M_R \times M_T$ MIMO system at any time instant can be represented by the following equation.
\begin{equation}
	Y_i = H_iX_i + N_i, i \in [1, S_c]
\end{equation}
In the equation above, the $M_R \times M_T$ dimensional channel matrix $H_i$ represents the Channel State Information (CSI) for the subcarrier $i$. Any two communicating WiFi devices estimate this channel matrix $H_i$ for every subcarrier by regularly transmitting a known preamble of OFDM symbols between each other \cite{ali2015keystroke}. For each Tx-Rx antenna pair, we can attain CSI values for $S_c$ = 30 OFDM subcarriers of the 20 MHz WiFi Channel by using CSI Tools and Intel 5300 network interface card. This leads to 30 matrices with dimensions $M_R \times M_T$ per CSI sample.
\subsection{Related Work}
There are a myriad of work for device-free gesture recognition. Some are based on Received Signal Strength Indicator (RSSI), some on Channel State Information (CSI).

\cite{pu2013whole} claims itself to be the first to do gesture recognition using wireless signals. 
They presented a system named WiSee, which extracts minute Doppler shifts from wide-band OFDM transmissions to enable whole-home sensing and recognition of human gestures.
There is another work \emph{Allsee} \cite{kellogg2014bringing} which leverages wireless signals furthermore and applies the method into RFID tags and power-harvesting sensors.

Another category is RSSI-based gesture recognition, which leverages the signal strength changes caused by human motion.
However, due to the low resolution of RSSI values provided by commercial devices, the performance of these kinds of methods is usually considered relatively low.
For instance, the accuracy is 56\% over 7 different gestures in \cite{sigg2014telepathic}.
Nevertheless, \cite{abdelnasser2015wigest} claims that they achieve 96\% when evaluating the system using a multi-media player application.
they also claim that they only need standard WiFi device, unlike \emph{Wisee} which uses USRP in their experiment.


\section{System Design} \label{section-design}
\subsection{Overview}
\subsection{Data Collection}
\subsection{Data Processing}
\subsection{Model Training}

\section{Implementation} \label{section-implementation}

\section{Evaluation} \label{section-evaluation}


\section{discussion} \label{section-discussion}

In this section, we discuss what we learned from this project, in which division of work and future work are also covered.

\subsection{Division of Work}
Yuning Mao is the team leader. He took great efforts to the research and survey of papers in which CSI is used. He installed the hardware and set up the environment of CSI tool in Ubuntu 12.04. He tried things first and assigned tasks to the team members after having a basic understanding. He modified the program in CSI tool which was originally used to save CSI into binary file into a program which was then used to do gesture recognition. He is also responsible for most experiments and batch files.

Yuting Jia is in charge of the model, including feature selection, model selection, model training and so on. He also rewrote the C file which was originally called within \emph{matlab} in the CSI tool so that we could leverage the CSI directly.

Zhenfeng Shi did the work of signal processing. He tried multiple methods of signal filtering and implemented them in both \emph{matlab} and \emph{python}.

\subsection{Limitations and Future Work}
Currently our system depends on the environment heavily and is sensitive to noise. In addition, we have to train different parameters of model in different scenarios. We think that there are a couple of reasons accounting for that. 
First, we couldn't find a quite and stable environment for data collection. The training data are hence with great noise, which is apparently undesirable. 
Second, the samples we collected are far from enough compared to the researchers who have been working on this field for years. For instance, in \cite{ali2015keystroke}, they obtained $10 persons \times 30samples/person/classes \times 37classes = 11100$ samples. We only have around one percent of them in each experiment. After all we had to collect data after establishing the whole system, and there was only one week or so and the exams were around the corner...
Future work directions are as follows. First, We need to collect more training data in a relatively stable environment. Second, we should try more advanced filtering methods via which we can hopefully eliminate random noise. Furthermore, we can try different models and different applications such as typing recognition and voice recognition, which are more challenging and exciting in the meantime!

\section{conclusion} \label{section-conclusion}
In this project, we did various experiments by leveraging channel state information (CSI) of WiFi signals using \emph{CSI Tool} and Intel 5300 NIC.
We install the Intel 5300 NIC into a lenovo R400 laptop and use external antennas to receive signals better.
We process the CSI and use SVM to train models to classify four different gestures and acquire an average accuracy of ??\%.
In addition, we use CSI to localize tables in the lab and achieve an average accuracy of ??\%.
It's true that there are still flaws and limits in our system. But We indeed started from scratch without instructions from the experienced and did plenty of experiments.

\section*{Acknowledgment}
We want to say thanks to Zhenyu Song, who let us know the CSI Tool and provide us with Intel 5300 NIC and laptop Lenovo R400. We also thank Prof.Jia for the WiFi router and instruction.

\bibliographystyle{IEEEtran}
\bibliography{IEEEabrv,repo}
% that's all folks
\end{document}


